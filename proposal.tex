%
% Proposal to The Unicode Consortium to include IEC-5007, -8, -9,
% -10, and IEEE 1621 "Stand-by" sumbols in Unicode.
%
% For information on this file please contact Joe Loughry at
% Tel. +1 303 221 4380 (time zone GMT minus 7 hours) or Email:
% joe.loughry@stx.ox.ac.uk
%

\documentclass[10pt,letterpaper]{article}

\usepackage[english,british]{babel}

% to let me use the new TrueType font we created:
\usepackage{fontspec}

% for formatting URLs (the `obeyspaces' option is for URLs with spaces (%20) in them):
\usepackage[obeyspaces]{url}

% added 20130901.2347 for putting angled brackets around URLs:
\newcommand{\URL}[1]{$\langle$\url{#1}$\rangle$}

% make references and footnotes into clickable links in the PDF:
\usepackage{hyperref}

% for XeLaTeX logo
\usepackage{hologo}

% Use \IEC{x} to display a symbol in the new font.
\newcommand{\IEC}[1]{{\fontspec{IECpower}#1}}

% Tell Adobe Acrobat Reader not to display the bookmarks pane.
\hypersetup{pdfpagemode=UseNone}

% Tell it to start in my preferred size.
\hypersetup{pdfstartview=FitH}

% Fill in some other information in the PDF header.
\hypersetup{pdftitle=Proposal to Include IEC Power Button Symbols}
\hypersetup{pdfauthor=Joe Loughry and Terence Eden}
\hypersetup{pdfsubject=Unicode Character Proposal}
\hypersetup{pdfkeywords={IEC power symbol Unicode}}

\begin{document}

\title{Proposal to Include IEC Power Button Symbols}

\author{Joe Loughry and Terence Eden}

\maketitle

\begin{abstract}
The international symbol IEC 60417-5009 \IEC{P} meaning `power' is not today
in Unicode. Clearly it would be useful to anyone writing technical or user
manuals. Furthermore, for electronically published documentation, it is
crucial for this and a few other symbols to be defined because it makes them
searchable in plain text. In this proposal we provide a TrueType font named
`IECpower' containing the glyphs as specified in three international standards
together with all of the needed character properties for Unicode specification.
\end{abstract}

\section{Introduction}

The \IEC{P}, \IEC{T}, \IEC{0}, and \IEC{1} symbols are defined in
IEC 60417 \cite{IEC60417}, which is also ISO 7000:2012 \cite{ISO7000}.
IEEE 1621 further defines \IEC{S} and refines the definition of \IEC{P}, notably
because IEEE 1621 standardises current practice as universally used on devices
with regard to the \IEC{P} symbol and to a lesser extent for the \IEC{S} symbol
as well \cite{IEEE1621}.

These characters, particularly \IEC{P}, are needed for technical writing and
are not in Unicode. The advantage of having them there would be search-ability
in plain text, something not possible with graphical symbols, as technical
writers are limited to using today.

We provide with our proposal a TrueType font, hereby released for unlimited use.

\subsection{The \emph{IEC Power} TrueType Font}

The five symbols included in the \emph{IEC Power} TrueType font are shown in
Table~\ref{table:symbols}. Only these symbols exist in the font; if an undefined
character, for example `A' is called for, the result is
implementation-defined.\footnote{In \hologo{XeTeX}, for example, the result of
`A' in \emph{IEC Power} is \IEC{A}.  In OpenOffice Writer, the result is the
letter `A' but in a san-serif typeface.}

\begin{table}[htbp]
	\centering
	\begin{tabular}{clcll}
		\textbf{Symbol} & \textbf{Applicable} & \textbf{Character} & \
			\textbf{Mnemonic} & \textbf{Meaning} \\
		& \textbf{Standard(s)} & \textbf{To Type} \\
		\hline \\
		\IEC{P} & IEC 60417-5009 & P & `power'       & Power        \\
		\IEC{T} & IEC 60417-5010 & T & `toggle'      & Power on/off \\
		\IEC{0} & IEC 60417-5008 & 0 & `binary zero' & Power off    \\
		\IEC{1} & IEC 60417-5007 & 1 & `binary one'  & Power on     \\
		\IEC{S} & IEEE 1621      & S & `sleep'       & Stand-by     \\
    \end{tabular}
    \caption{All of the available glyphs in the \emph{IEC power} TrueType font.}
    \label{table:symbols} % label must come after caption!
\end{table}

Placement of the symbols in the \emph{IEC Power} TrueType font was chosen
thoughtfully so as to be mnemonic: `P' for power, `S' for stand-by or sleep,
`T' for toggling power on or off, and `1' and `0' for power-on and power-off,
respectively, all of these mnemonics `fail gracefully' should the \emph{IEC Power}
font happen to be unavailable.

\subsection{Example Usage}

In-line in text with normal spacing, the \IEC{P} characters \IEC{S} look \IEC{T}
like \IEC{1} this \IEC{0}.

\section{Character Properties}

Suggested character properties for the proposed symbols are given in Table
\ref{table:character-properties}.

\begin{table}[htbp]
	\centering
	\begin{tabular}{ll}
		\textbf{Property} & \textbf{Value} \\
		\hline \\
		A & B \\
    \end{tabular}
    \caption{Suggested character properties.}
    \label{table:character-properties} % label must come after caption!
\end{table}

\section{Anticipated Objections}

It might be argued that the meaning of \IEC{P} is disputed between IEC 60417
and IEEE 1621, {\it i.e.}, that IEC 60417 (as well as ISO 7000:2012) defined \IEC{P}
to mean `stand-by' and IEEE 1621 changed it to mean `power'. We counter that
the issue is irrelevant to the Unicode Consortium for two reasons: firstly,
because the symbol itself is needed by writers, who could use it (if it were
available to them) according to local conventions regardless of the disagreement
between IEC/ISO and IEEE; and secondly, because IEEE 1621 specifically codifies
\emph{existing practice}; the number of devices out there using \IEC{P} to mean
`power' dwarfs the number of devices that use it to mean `stand-by' \cite{IEEE1621}.

There are, of course, many characters in Unicode already resembling circles
(\IEC{0}), or lines (\IEC{1}), or the crescent moon (\IEC{S}). None of the existing
characters, however, has anything semantically to do with the concepts of `power',
`switch', `toggle', or `interrupter'. There are eleven occurrences of the word `power'
in Version 6.3.0 of the Unicode standard (Table \ref{table:power}) but none has
anything to do with device control \cite{Unicode2013}.

The proposed characters are not part of any script and the precise form of
their drawing is not critical. 

\begin{table}[htbp]
	\centering
	\begin{tabular}{lcl}
		\textbf{Section} & \textbf{Code Point} & \textbf{Description} \\
		\hline
		Telugu fractions \
			& 0C78 & TELUGU FRACTION DIGIT ZERO \\
		and weights	& & FOR ODD POWERS OF FOUR \\
			& 0C79 & TELUGU FRACTION DIGIT ONE \\
				& & FOR ODD POWERS OF FOUR \\
			& 0C7A & TELUGU FRACTION DIGIT TWO \\
				& & FOR ODD POWERS OF FOUR \\
			& 0C7B & TELUGU FRACTION DIGIT THREE \\
				& & FOR ODD POWERS OF FOUR \\
			& 0C7C & TELUGU FRACTION DIGIT ONE \\
				& & FOR EVEN POWERS OF FOUR \\
			& 0C7D & TELUGU FRACTION DIGIT TWO \\
				& & FOR EVEN POWERS OF FOUR \\
			& 0C7E & TELUGU FRACTION DIGIT THREE \\
				& & FOR EVEN POWERS OF FOUR \\
		\hline
		Miscellaneous \
			& 26EE & GEAR WITH HANDLES \\
		Symbols & & (= power plant, power substation) \\
		\hline
		Kangxi Radicals \
			& 2F12 & KANGXI RADICAL POWER \\
		\hline
		Yijing Hexagram Symbols \
			& 4DE1 & HEXAGRAM FOR GREAT POWER \\
		\hline
		Mathematical \
			& 1D4AB & MATHEMATICAL SCRIPT \\
		Alphanumeric Symbols & & CAPITAL P (= power set) \\
    \end{tabular}
    \caption{All occurrences of `power' in the Unicode Standard, Version 6.3.0.}
    \label{table:power} % label must come after caption!
\end{table}

\subsection{Sever-ability}

Of all the characters in Table \ref{table:symbols}, the most needed is \IEC{P}.
We included the others in this proposal because they form a logical group. If,
however, there is any objection to inclusion of \IEC{1}, \IEC{0}, \IEC{T}, or
\IEC{S}, the one we really need is \IEC{P}.

\section{Summary and Conclusion}

The \IEC{P} and \IEC{S} symbols are most important because nothing like them
appears already in Unicode.

\ldots

\bibliographystyle{plain}
\bibliography{consolidated_bibtex_file}

\vfill
{\tiny Build \input{build_counter.txt}}

\end{document}

