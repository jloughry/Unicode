%
% Proposal to The Unicode Consortium to include IEC-5007, -8, -9,
% -10, and IEEE 1621 "Stand-by" sumbols in Unicode.
%
% For information on this file please contact Joe Loughry at
% Tel. +1 303 221 4380 (time zone GMT minus 7 hours) or Email:
% joe.loughry@stx.ox.ac.uk
%

\documentclass[10pt,a4paper]{article}

\usepackage[english,british]{babel}

% to let me use the new TrueType font we created:
\usepackage{fontspec}

% for formatting URLs (the `obeyspaces' option is for URLs with spaces (%20) in them):
\usepackage[obeyspaces]{url}

% added 20130901.2347 for putting angled brackets around URLs:
\newcommand{\URL}[1]{$\langle$\url{#1}$\rangle$}

% make references and footnotes into clickable links in the PDF:
\usepackage{hyperref}

% for XeLaTeX logo
\usepackage{hologo}

% Use \IEC{x} to display a symbol in the new font.
\newcommand{\IEC}[1]{{\fontspec{IECpower}#1}}

% Tell Adobe Acrobat Reader not to display the bookmarks pane.
\hypersetup{pdfpagemode=UseNone}

% Tell it to start in my preferred size.
\hypersetup{pdfstartview=FitH}

% Fill in some other information in the PDF header.
\hypersetup{pdftitle=Proposal to Include IEC Power Button Symbols}
\hypersetup{pdfauthor=Joe Loughry and Terence Eden}
\hypersetup{pdfsubject=Unicode Character Proposal}
\hypersetup{pdfkeywords={IEC power symbol Unicode}}

\begin{document}

\title{Proposal to Include IEC Power Button Symbols}

\author{Joe Loughry and Terence Eden \\
\\
\URL{mailto:joe.loughry@gmail.com}}

\maketitle

\begin{abstract}
The international symbol IEC 60417-5009 \IEC{P} meaning `power' is not in
Unicode. Clearly it would be useful to anyone writing technical or user
manuals. Furthermore, for electronically published documentation, it is
crucial for this and a few other symbols to be defined because it makes them
searchable in plain text. In this proposal we provide a TrueType font named
`IECpower' containing the glyphs as specified in three international standards
together with all of the needed character properties for Unicode specification.
\end{abstract}

\section{Introduction}

The \IEC{P}, \IEC{T}, \IEC{0}, and \IEC{1} symbols are defined in
IEC 60417 \cite{IEC60417}, which is also ISO 7000:2012 \cite{ISO7000}.
IEEE 1621 defines \IEC{S} and refines the definition of \IEC{P}, notably
by saying:

\begin{quote}
IEC 60417 defines \IEC{P} for use with a power switch that does not do a total
mains disconnect, and hence the device consumes standby power. \IEC{P} is
generally used and understood to mean ``power,'' as on power buttons, indicators,
and elsewhere. \IEC{P}, therefore, means ``power'' with a nonzero power level
in the \emph{off} state. Electronic devices shall use \IEC{P} to be a synonym
for ``power'' on power controls.
\end{quote}

\noindent \cite[\S 4.3, emphasis in original]{IEEE1621}. IEEE 1621 standardises
current practice for devices with regard to the \IEC{P} symbol and to a lesser
extent for \IEC{S}.

These characters, particularly \IEC{P}, are needed for technical writing and
are not in Unicode. The advantage of having them there would be that for the
first time they would be searchable in plain text, something not possible with
embedded graphics, which is the way the symbols have been displayed to date.

\section{Suitability for Inclusion}

These symbols are \emph{characters} according to the definition in the Glossary,
and do not appear in the Archive of Notices of Non-Approval. They are neither
in the Pipeline Table nor in BETA. They are all widely used on equipment and
would benefit technical writers if they were available in Unicode and benefit
readers because it would make user manuals and other technical documentation
more searchable in plain text.

We provide with our proposal a TrueType font, with no restrictions on its use.

\section{Character Properties}

Suggested character properties for the proposed symbols are given in Tables
\ref{table:character-properties-P}--\ref{table:character-properties-S}. These
are the same names as in IEEE 1621-2004. None of the proposed names appear
already in the Character Name Index.

\begin{table}[htbp]
	\centering
	\begin{tabular}{ll}
		\textbf{Property} & \textbf{Suggested Value} \\
		\hline \\
		Code point                & \emph{to be determined} \\
		Name                      & POWER \\
		General Category          & So \\
		Canonical Combining Class & 0 \\
		Bidirectional Class       & ON \\
		Decomposition Type/Decomposition Mapping \\
		Numeric Type \\
		Numeric Value \\
		Bidi Mirrored             & N \\
		Unicode 1 Name \\
		ISO Comment \\
		Simple Uppercase Mapping \\
		Simple Lowercase Mapping \\
		Simple Titlecase Mapping \\
    \end{tabular}
    \caption{Suggested character properties for \IEC{P}.}
    \label{table:character-properties-P} % label must come after caption!
\end{table}

\begin{table}[htbp]
	\centering
	\begin{tabular}{ll}
		\textbf{Property} & \textbf{Suggested Value} \\
		\hline \\
		Code point                & \emph{to be determined} \\
		Name                      & ON \\
		General Category          & So \\
		Canonical Combining Class & 0 \\
		Bidirectional Class       & ON \\
		Decomposition Type/Decomposition Mapping \\
		Numeric Type \\
		Numeric Value \\
		Bidi Mirrored             & N \\
		Unicode 1 Name \\
		ISO Comment \\
		Simple Uppercase Mapping \\
		Simple Lowercase Mapping \\
		Simple Titlecase Mapping \\
    \end{tabular}
    \caption{Suggested character properties for \IEC{1}.}
    \label{table:character-properties-1} % label must come after caption!
\end{table}

\begin{table}[htbp]
	\centering
	\begin{tabular}{ll}
		\textbf{Property} & \textbf{Suggested Value} \\
		\hline \\
		Code point                & \emph{to be determined} \\
		Name                      & OFF \\
		General Category          & So \\
		Canonical Combining Class & 0 \\
		Bidirectional Class       & ON \\
		Decomposition Type/Decomposition Mapping \\
		Numeric Type \\
		Numeric Value \\
		Bidi Mirrored             & N \\
		Unicode 1 Name \\
		ISO Comment \\
		Simple Uppercase Mapping \\
		Simple Lowercase Mapping \\
		Simple Titlecase Mapping \\
    \end{tabular}
    \caption{Suggested character properties for \IEC{0}.}
    \label{table:character-properties-0} % label must come after caption!
\end{table}

\begin{table}[htbp]
	\centering
	\begin{tabular}{ll}
		\textbf{Property} & \textbf{Suggested Value} \\
		\hline \\
		Code point                & \emph{to be determined} \\
		Name                      & ON/OFF \\
		General Category          & So \\
		Canonical Combining Class & 0 \\
		Bidirectional Class       & ON \\
		Decomposition Type/Decomposition Mapping \\
		Numeric Type \\
		Numeric Value \\
		Bidi Mirrored             & N \\
		Unicode 1 Name \\
		ISO Comment \\
		Simple Uppercase Mapping \\
		Simple Lowercase Mapping \\
		Simple Titlecase Mapping \\
    \end{tabular}
    \caption{Suggested character properties for \IEC{T}.}
    \label{table:character-properties-T} % label must come after caption!
\end{table}

\begin{table}[htbp]
	\centering
	\begin{tabular}{ll}
		\textbf{Property} & \textbf{Suggested Value} \\
		\hline \\
		Code point                & \emph{to be determined} \\
		Name                      & SLEEP \\
		General Category          & So \\
		Canonical Combining Class & 0 \\
		Bidirectional Class       & ON \\
		Decomposition Type/Decomposition Mapping \\
		Numeric Type \\
		Numeric Value \\
		Bidi Mirrored             & N \\
		Unicode 1 Name \\
		ISO Comment \\
		Simple Uppercase Mapping \\
		Simple Lowercase Mapping \\
		Simple Titlecase Mapping \\
    \end{tabular}
    \caption{Suggested character properties for \IEC{S}.}
    \label{table:character-properties-S} % label must come after caption!
\end{table}

\subsection{Collation Order}\label{section:collation-order}

There is no required collation order, although there is an implied state transition
ordering:

\begin{quote}
Power states shall be understood to have physical relationships to each other.
Specifically, \emph{on} is taken to be above \emph{sleep}, and \emph{sleep} above
\emph{off}.
\end{quote}

\noindent \cite[\S 4.4, emphasis in original]{IEEE1621}. They exhibit no shaping
behaviour and have no particular required sorting order (except see the quoted
paragraph above). The characters are uncased. There is no special
linebreaking behaviour required. These characters are not meant for use in
identifiers, although they have been used for such. They are stand-alone symbols.
They are not white-space characters and have no numeric values. They are neither
combining characters nor punctuation.

\section{The \emph{IECpower} TrueType Font}

The five symbols included in the \emph{IECpower} TrueType font are shown in
Table~\ref{table:symbols}. Only these symbols exist in the font; if an undefined
character, for example `A' is called for, the result is
implementation-defined.\footnote{In \hologo{XeTeX}, for example, the result of
`A' in \emph{IECpower} is \IEC{A}.  In OpenOffice Writer, the result is the
letter `A' but in a san-serif typeface.}

\begin{table}[htbp]
	\centering
	\begin{tabular}{clcll}
		\textbf{Symbol} & \textbf{Applicable} & \textbf{Character} & \
			\textbf{Mnemonic} & \textbf{Meaning} \\
		& \textbf{Standard(s)} & \textbf{To Type} \\
		\hline \\
		\IEC{P} & IEC 60417-5009 & P & `power'       & Power        \\
		\IEC{T} & IEC 60417-5010 & T & `toggle'      & Power on/off \\
		\IEC{0} & IEC 60417-5008 & 0 & `binary zero' & Power off    \\
		\IEC{1} & IEC 60417-5007 & 1 & `binary one'  & Power on     \\
		\IEC{S} & IEEE 1621      & S & `sleep'       & Stand-by     \\
    \end{tabular}
    \caption{All of the available glyphs in the \emph{IEC power} TrueType font.}
    \label{table:symbols} % label must come after caption!
\end{table}

Placement of the symbols in the \emph{IECpower} TrueType font was chosen
thoughtfully so as to be mnemonic: `P' for power, `S' for stand-by or sleep,
`T' for toggling power on or off, and `1' and `0' for power-on and power-off,
respectively; these mnemonics `fail gracefully' in text should the
\emph{IECpower} font happen to be unavailable.

\subsection{Example Usage}

In-line in text with normal spacing, the \IEC{P} characters \IEC{S} look \IEC{T}
like \IEC{1} this \IEC{0}.

\section{Anticipated Objections}

It might be argued that the meaning of \IEC{P} is disputed between IEC 60417
and IEEE 1621, {\it i.e.}, that IEC 60417 (as well as ISO 7000:2012) defined \IEC{P}
to mean `stand-by' and IEEE 1621 changed it to mean `power'. We counter that
the issue is irrelevant to the Unicode Consortium for two reasons: firstly,
because the symbol itself is needed by writers, who could use it---if it were
available to them---according to local conventions regardless of the disagreement
between IEC/ISO and IEEE; and secondly, because IEEE 1621 specifically codifies
existing practice; the number of devices out there using \IEC{P} to mean
`power' dwarfs the number of devices that use it to mean `stand-by'. Furthermore,

\begin{quote}
No safety issue is introduced by the use of the symbol on a switch that causes
the device to go to a \emph{hard-off} state.
\end{quote}

\noindent \cite[\S 4.3, emphasis in original]{IEEE1621}.

There are, of course, many characters in Unicode already resembling circles
(\IEC{0}), or lines (\IEC{1}), or the crescent moon (\IEC{S}). None of the existing
characters, however, has anything semantically to do with the concepts of `power',
`switch', `toggle', or `interrupter'. There are several occurrences of the crescent
moon, but none showing the \IEC{S} phase. There are eleven occurrences of the word
`power' in Version 6.3.0 of the Unicode standard (Table \ref{table:power}) but none
has anything to do with device control \cite{Unicode2013}.

\section{Drawing the Symbols}

The proposed characters are not part of any script and the precise form of
their drawing is not critical. As IEEE 1621-2004 says:

\begin{quote}
In accordance with IEC 80416-3, symbols can be filled, be rotated, have their lines
thickened, or be used on digital displays, as long as an ordinary user can recognize
the symbol correctly.
\end{quote}

\noindent \cite[\S 4.3]{IEEE1621}.

\begin{table}[htbp]
	\centering
	\begin{tabular}{lcl}
		\textbf{Section} & \textbf{Code Point} & \textbf{Description} \\
		\hline \\
		Telugu fractions \
			& 0C78 & TELUGU FRACTION DIGIT \\
		and weights	& & ZERO FOR ODD POWERS \\
				& & OF FOUR \\
			& 0C79 & TELUGU FRACTION DIGIT \\
				& & ONE FOR ODD POWERS \\
				& & OF FOUR \\
			& 0C7A & TELUGU FRACTION DIGIT \\
				& & TWO FOR ODD POWERS \\
				& & OF FOUR \\
			& 0C7B & TELUGU FRACTION DIGIT \\
				& & THREE FOR ODD POWERS \\
				& & OF FOUR \\
			& 0C7C & TELUGU FRACTION DIGIT \\
				& & ONE FOR EVEN POWERS \\
				& & OF FOUR \\
			& 0C7D & TELUGU FRACTION DIGIT \\
				& & TWO FOR EVEN POWERS \\
				& & OF FOUR \\
			& 0C7E & TELUGU FRACTION DIGIT \\
				& & THREE FOR EVEN \\
				& & POWERS OF FOUR \\
		\hline
		Miscellaneous \
			& 26EE & GEAR WITH HANDLES \\
		Symbols & & (= power plant, power \\
				& & substation) \\
		\hline
		Kangxi Radicals \
			& 2F12 & KANGXI RADICAL POWER \\
		\hline
		Yijing Hexagram Symbols \
			& 4DE1 & HEXAGRAM FOR GREAT \\
				& & POWER \\
		\hline
		Mathematical \
			& 1D4AB & MATHEMATICAL SCRIPT \\
		Alphanumeric Symbols & & CAPITAL P (= power set) \\
    \end{tabular}
    \caption{All occurrences of `power' in the Unicode Standard, Version 6.3.0.}
    \label{table:power} % label must come after caption!
\end{table}

\subsection{Sever-ability}

Of all the characters in Table \ref{table:symbols}, the most needed is \IEC{P}.
We included the others in this proposal because they form a logical group. If,
however, there is any objection to inclusion of \IEC{1}, \IEC{0}, \IEC{T}, or
\IEC{S}, the one we really need is \IEC{P}.

\section{Sponsor}

The postal address for correspondence is:

\begin{quote}
Joe Loughry \\
6214 South Krameria Street \\
Centennial, Colorado 80111-4243 \\
USA
\end{quote}

Tel.\ +1 303 221 4380 (time zone US/Mountain, GMT minus 7 hours).

\section{Summary and Conclusion}

The \IEC{P}, \IEC{0}, \IEC{1}, \IEC{T}, and \IEC{S} symbols are needed by
technical writers to produce manuals in which these important symbols are
searchable in plain text. Because they were invented by the standards body to be
distinctive, new, and unambiguous, there is no confusion with existing scripts.
The suggested character properties are simple. We provide along with this
proposal a TrueType font called \emph{IECpower} containing the new symbols;
the TrueType font is made available with no restrictions.

\bibliographystyle{plain}
\bibliography{consolidated_bibtex_file}

\vfill
{\tiny Build \input{build_counter.txt}}

\end{document}

