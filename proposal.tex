%
% Proposal to The Unicode Consortium to include IEC-5007, -8, -9,
% -10, and IEEE 1621 "Stand-by" sumbols in Unicode.
%
% For information on this file please contact Joe Loughry at
% Tel. +1 303 221 4380 (time zone GMT minus 7 hours) or Email:
% joe.loughry@stx.ox.ac.uk
%

\documentclass[10pt,letterpaper]{article}

\usepackage[english,british]{babel}
\usepackage[pdftex]{graphicx}
\usepackage[T1]{fontenc}

\newcommand\customfont[1]{{\usefont{T1}{iec-unicode-font}{m}{n} #1 }}

\begin{document}

\title{Proposal to Include IEC Power Button Symbols in Unicode}

\author{Joe Loughry and Terence Eden}

\maketitle

\begin{abstract}
The international symbols \customfont{P} meaning `power on' and \customfont{P}
meaning `stand-by' are not in Unicode.  Clearly these would be useful to anyone
writing technical or user manuals. Furthermore, for electronically published
documentation, it is crucial to have the symbols defined in Unicode because it
makes them search-able in text.  In this proposal we provide a TrueType font
containing the glyphs as specified in three international standards together
with all needed character properties.
\end{abstract}

\section{Introduction}

It has long been known\ldots \cite{Kafka1946}

\bibliographystyle{plain}
\bibliography{consolidated_bibtex_file}

\end{document}

